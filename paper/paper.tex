% vim: set spell spelllang=en tw=100 et sw=4 sts=4 :

\documentclass{article}
\usepackage{ijcai17}
\usepackage{times}
\usepackage{microtype}
\usepackage{amsmath}

% \usepackage{showframe}

\title{Value-Ordering, Discrepancies, and Restarts in Subgraph Solvers: \\ Some Half-Baked Mucking
Around}
\author{Ciaran McCreesh\thanks{This work was supported by the Engineering and Physical Sciences
    Research Council [grant number EP/026842/1]}\\ University of Glasgow, Glasgow, Scotland \\
    ciaran.mccreesh@glasgow.ac.uk}

\begin{document}

\maketitle

\begin{abstract}
    Modern subgraph isomorphism solvers use reasonably good value-ordering heuristics to direct
    search, but they are not perfect, particularly for early choices. We investigate two mechanisms
    for offsetting this weakness: discrepancy searches, and restarts. Our results suggest that
    discrepancy search is too expensive to be effective, but that restarts with a random
    value-ordering heuristic beat tailored heuristics. We then design a slightly-random
    value-ordering heuristic which, when combined with restarts, performs ?? times better on
    satisfiable instances without substantially weakening performance on unsatisfiable instances.
\end{abstract}

\section{Introduction}

?? Subgraph isomorphism is useful.

The current strongest subgraph isomorphism solver uses degree as a value-ordering heuristic. This is
much better than not having a value-ordering heuristic, but it is far from perfect, particularly for
early branching choices. There are two reasons we suspect we can do better. Firstly, . Secondly, even if degree
were a perfect source of information, many graphs do not have a large degree distribution, and so we
still need to determine \emph{which} vertex of highest degree would be best.

We could try to introduce tiebreaking mechanisms. However, given the extreme cost of making an
incorrect branching choice early on in search, we believe it is more fruitful to investigate
alternatives to simple backtracking (depth-first) search.

\section{The Basic Algorithm}

?? Started with the Glasgow algorithm, removed parts which did not contribute hugely to speedups,
and ended up with a simpler algorithm that performs as well in practice. In particular, large
graphs, no ILF, and not doing supplemental graphs of length 3.

\subsection{Ordering Heuristics}

?? Variable ordering is smallest domain first with degree tiebreaking. This is not the focus of this
paper.

In our experiments, we compare five value-ordering heuristics.

\begin{description}
    \item[Tailored] Where we select vertices from highest degree to lowest degree (which is done by
        permuting the graph at the top of search).
    \item[Anti-Tailored] Where we select vertices from lowest degree to highest degree.
    \item[Extra Tiebreaking] Where we select vertices from highest degree to lowest degree,
        tiebreaking on highest sum of neighbour degrees.
    \item[Random] Where we select a branching vertex at random by (dynamically, that is, we shuffle
        the vertices before entering the for loop, rather than by randomly permuting the graph once at the
        top of search).
    \item[Tailor-Weighted Random] Where we shuffle the vertices before each for loop, but instead of
        picking a vertex uniformly at random, we use a weight of
        $2^{\left|\operatorname{N}(v)\right|}$ when deciding whether to branch on vertex $v$
        next.\footnote{Actually, we cheat a bit to avoid range issues with random number generation:
        we give the vertex with highest degree in the entire graph weight $2^{50}$, and then give
        lower degree vertices smaller powers, giving every vertex weight at least 1. This is a bit
        of hack but getting it perfectly right is hard.} This scheme heavily prefer vertices of high
        degree, but will give the same chance of being selected next to two vertices of equal
        degree. This also introduces an element of randomness, which will be important for restarts.
\end{description}

\subsection{Discrepancy Searches}

?? Not dealing with binary search. Make it binary. Other versions where we increase the discrepancy
each time the degree changes, and where we count every ``against'' branch as one discrepancy.

\subsection{Restarts}

?? Aimed primarily at value-ordering, not value-ordering.

?? Value-ordering options

?? Propagation

?? We use the Luby scheme, because everyone else does. We count the number of backtracks (that is,
when we reach the end of the main for loop) to decide when to restart. Following established wisdom,
we multiply each item in the sequence by a magic constant. Preliminary experiments demonstrated that
this is useful, but we failed to determine a principled way of selecting the constant's value, so we
fell back on divine revelation and set it to 666. As with everyone else, we are not entirely clear
why we are doing this: it could be to allow the solver to spend more time deep in search, it could
be to reduce the overheads from repeatedly visiting inner nodes, or it could be to keep the nogoods
small.

?? Increasing nogoods etc. We are not confident enough in our programming abilities to implement
these correctly. Cannot propagate on negative decisions.

\section{Empirical Evaluation}

\subsection{Instances}

these, partly because some earlier solvers find many of these instances extremely hard, and partly
to avoid skewing a portfolio. We want larger numbers of satisfiable instances, so we include all
pattern / target pairs.

\subsection{Our Solver is Good}

?? Non-induced, compare to Glasgow, LAD, PathLAD, SND?, VF2

?? Induced, compare to Glasgow, LAD, PathLAD, VF2, VF3, McSplit.

\subsection{Discrepancies Don't Help}

?? DDS counting bias both ways, LDS

\subsection{Restarts Make it Better}

?? Random, tailored, antitailored, random with restarts, tailored biased, tailored biased with
restarts. Satisfiable and unsatisfiable split.

?? Breakdown by family

\subsection{Near the Phase Transition}

?? Closer look at SAT instances near the phase transition

\section{Other Subgraph Solvers}

?? Also do this in McSplit and kdown?

\section{Conclusion}

?? Parallel

\bibliographystyle{named}
\bibliography{dblp}

\end{document}

