% vim: set spell spelllang=en tw=100 et sw=4 sts=4 :

\documentclass{article}
\usepackage{ijcai17}
\usepackage{times}
\usepackage{microtype}

\usepackage{showframe}

\title{Value-Ordering, Discrepancies, and Restarts in Subgraph Solvers}
\author{Ciaran McCreesh\thanks{This work was supported by the Engineering and Physical Sciences
    Research Council [grant number EP/026842/1]}\\ University of Glasgow, Glasgow, Scotland \\
    ciaran.mccreesh@glasgow.ac.uk}

\begin{document}

\maketitle

\begin{abstract}
    Modern subgraph isomorphism solvers use reasonably good value-ordering heuristics to direct
    search, but they are not perfect, particularly for early choices. We investigate various forms
    of discrepancy search, but get best results by combining restarts with biased randomness in
    value-ordering heuristics.
\end{abstract}

\section{Introduction}

\section{The Basic Algorithm}

?? Started with the Glasgow algorithm, removed parts which did not contribute hugely to speedups,
and ended up with a simpler algorithm that performs as well in practice. In particular, large
graphs, no ILF, and not doing supplemental graphs of length 3.

\subsection{Ordering Heuristics}

\begin{description}
    \item[Random]
    \item[Tailored]
    \item[Anti-Tailored]
    \item[Tailor-Weighted Random]
\end{description}

\subsection{Discrepancy Searches}

?? Not dealing with binary search. Make it binary. Other versions where we increase the discrepancy
each time the degree changes, and where we count every ``against'' branch as one discrepancy.

\subsection{Restarts}

?? Aimed primarily at value-ordering, not value-ordering.

?? Value-ordering options

?? Propagation

?? We use the Luby scheme, because everyone else does. We count the number of backtracks (that is,
when we reach the end of the main for loop) to decide when to restart. Following established wisdom,
we multiply each item in the sequence by a magic constant. Preliminary experiments demonstrated that
this is useful, but we failed to determine a principled way of selecting the constant's value, so we
fell back on divine revelation and set it to 666. As with everyone else, we are not entirely clear
why we are doing this: it could be to allow the solver to spend more time deep in search, it could
be to reduce the overheads from repeatedly visiting inner nodes, or it could be to keep the nogoods
small.

?? Increasing nogoods etc. We are not confident enough in our programming abilities to implement
these correctly. Cannot propagate on negative decisions.

\section{Empirical Evaluation}

\subsection{Our Solver is Good}

?? Non-induced, compare to Glasgow, LAD, PathLAD, SND?, VF2

?? Induced, compare to Glasgow, LAD, PathLAD, VF2, VF3, McSplit.

\subsection{Discrepancies Don't Help}

?? DDS counting bias both ways, LDS

\subsection{Restarts Make it Better}

?? Random, tailored, antitailored, random with restarts, tailored biased, tailored biased with
restarts. Satisfiable and unsatisfiable split.

?? Breakdown by family

\subsection{Near the Phase Transition}

?? Closer look at SAT instances near the phase transition

\section{Other Subgraph Solvers}

?? Also do this in McSplit and kdown?

\section{Conclusion}

?? Parallel

\bibliographystyle{named}
\bibliography{dblp}

\end{document}

